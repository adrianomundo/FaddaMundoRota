
\documentclass {article}

\usepackage {graphicx}
\graphicspath {{./Images/}}
			

\begin{document}

\begin{figure}
\centering
	\includegraphics[height=8cm]{polimi_logo.png}
\end{figure}


\title {{\Huge \it SafeStreets} \\ \Large Software Engineering 2 Project}
\author{Salvatore Fadda, Adriano Mundo, Francesco Rota
		\\ \\ A.Y. 2019/2020 \\ Version 1.0}
\date{November 10, 2019}


\maketitle
\newpage

	
%Index	
\tableofcontents
\newpage


%Introduction
\section{Introduction}
The following RASD aims at providing an overview of the project {\it SafeStreets}. 

	
	\subsection{Purpose}

From this brief description of the functionalities we may extract the following goals for {\it SafeStreets}: 
		
		\begin{itemize} %GOALS

   			 \item {\bf [G1]} 			
			 \item {\bf [G2]} 
   			 \item {\bf [G3]}    			  
   			 \item {\bf [G4]} 
   			 \\
			 \\
With regards to {\it {\it Advanced Function 1}}, we may identify one goal: 
   			 \item {\bf [G6]}\\
			 \\
 With regards to {\it Adanced Function 2}, we extract the following goals:
			  \item {\bf [G7]} 			
			  \end{itemize}
			
	\subsection{Scope}
	dsdsdds
	\\		
		\subsubsection{World Phenomena}
		
We identify the following world phenomena:
		\begin{itemize}
			\item		
		\end{itemize}

		\subsubsection{Shared Phenomena}
		
		\subsubsection{Machine Phenomena}


	\subsection{Definitions, Acronyms, Abbreviations}
					
		\subsubsection{Definitions}
			
			\begin{itemize}
				\item {\bf Municipality:} 
				\item {\bf Violation:} the action of violating traffic 					laws
				\item {\bf Ticket:} administrative sanction established 					by law for a violation
				\item {\bf Plate recognition Algorithm:} algorithm that 					automatic recognize veicles' plate by the images sent 						from users to report a violation or an accident 
				\item {\bf Notification Data:} infromation that is 						povided by the user when he reports a violation. This 						includes picture, license plate, date, time, position.
				\item {\bf Ticket Data:} infromation that is povided by 					the municipality when it adds a new ticket in the 							system. This includes violation type, license plate, 						date, time, position.
				\item {\bf Accident Data:} infromation that usually is 					povided by the municipality when it reports an accident. 				This can includes accident type,picture, multiple 							license plate, date, time, position.
				\item {\bf Intervention:} action taken by the 								municipality to prevent further issues in the city 						traffic.
				\item {\bf Notification:} message sent by the user to 						advise the system about a violation.
			\end{itemize}
		
		\subsubsection{Acronyms}
		
			\begin{itemize}
				\item {\bf GPS:} Global Positioning System
				\item {\bf API:} Aplication Programming Interface
				\item {\bf ID:} Identifier
				\item {\bf RASD:} Requirements Analysis and 								Specification Document
				\item {\bf DBMS:} Batabase Management System
				\item {\bf GDPR:} General Data Protection Regulation
			\end{itemize}

				
				
		\subsubsection{Abbreviations}
		
			\begin{itemize}
				\item {\bf Gn:} n-th goal
				\item {\bf Rn:} n-th functional requirement
				\item {\bf Dn:} n-th domain assumption
				\item {\bf AF1:} advanced function one
				\item {\bf AF2:} advanced function two
				\item {\bf SP1:} shared phenomena controlled by the 						World and observed by the Machine
				\item {\bf SP2:} shared phenomena controlled by the 						Machine and observed by the World
				\item {\bf WP:} World Phenomena
				\item {\bf MP:} Machine Phenomena
			\end{itemize}
			

	\subsection{Revision History}
	
	\begin{table}[ht]
		\centering
		\begin{tabular}{ccc} 
		Version & Date & Changes  \\ 
		\hline
		 \\
		 \\
		\end{tabular}
		\caption{Revision History}
		\label{default}
	\end{table}
	
	\subsection{Reference Documents}
		
		\begin{itemize}
   	
			 \item Project Assignment
			\end{itemize}

	\subsection{Document Structure}
			The rest of the document is organized as follows:
				\begin{itemize}
					\item {\bf Overall Description} (Section 2) 					
					\item {\bf Specific Requirements} (Section 3) 
					\item {\bf Formal Analysis} (Section 4)				
					\end{itemize}

\pagebreak	
	
%Overall Description
\section{Overall Description}
	\subsection{Product perspective}
	
	\subsubsection{Class Diagrams}
		

	\subsection{Product functions}
		
	\subsection{User characteristics}
	
	\subsection{Assumptions, dependencies and constraints}
		\subsubsection{Constraints}
		\begin{itemize}
		\item		
		\end{itemize}
		\subsubsection{Dependencies}
		\begin{itemize}
		\item 
		\end{itemize}
		
		\subsubsection{Domain Assumptions}
		\begin{itemize}
		 \item {\bf [D1]}  
		 \item 		\end{itemize}
		
\pagebreak	


	
\end{document}