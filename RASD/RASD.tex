
\documentclass {article}

\usepackage {graphicx}
\graphicspath {{./Images/}}
			

\begin{document}

\begin{figure}
\centering
	\includegraphics[height=8cm]{polimi_logo.png}
\end{figure}


\title {{\Huge \it SafeStreets} \\ \Large Software Engineering 2 Project - Prof. Matteo Rossi}
\author{Salvatore Fadda, Adriano Mundo, Francesco Rota
		\\ \\ A.Y. 2019/2020 \\ Version 1.0}
\date{November 10, 2019}



\maketitle
\newpage

	
%Index	
\tableofcontents
\newpage


%Introduction - Section 1
\section{Introduction}
This document represents the {\it Requirements Analysis and Specification Document} (RASD). It aims at providing an overview of the project {\it SafeStreets}. It illustrates the purpose of the project, starting from its goals and how these can be reached with certain non functional requirements, functional requirements and constraints. This document is intended for all the people involved in the project life-cycle, from planning and estimation to development and validation.

	
	\subsection{Purpose}
	
	{\it SafeStreets} is a crowd-sourced application that aims at keeping safe the city's streets. The application goal is to allow {\it Users} to notify the {\it Municipality} when a violation occur on the streets under its jurisdiction. The {\it User} can notice and notify the violation by sending a photo of the violation including date, time and position. These violations are for the majority parking violation, such as double parking or vehicle parked in the middle of bike lanes. \\ \\
	{\it SafesStreets}, once the {\it User} has notified the violation, stores all the data, completing them with all the necessary metadata. In order to be sure that the violation is correctly elaborated, the application uses a plate recognition algorithm. The {\it User} is notified if something goes wrong during the whole process, so an alternative solution can be found. All the data stored by {\it SafeStreets} are provided by the {\it Users} or they can be retrieved directly from the device, like the position from the GPS system.\\
	All the data collected by {\it SafeStreets} can be mined by {\it Users} and the {\it Authorities}, so they can be provided with statistics built from this data. The application can show different statistics based on different level of visibility, so the {\it Authorities} can access to some information that the {\it Users} can not access, and viceversa. This is the purpose of the {\it Basic Service}. \\ \\
	The application has also two specific {\it Advanced Functions}. In the {\bf AF1}, the application {\it SafeStreets} can cross the information about the road accident, thanks to a service offered by the {\it Municipality}, with the data stored and retrieved from the {\it User}. So {\it SafeStreets } elaborates them in order to identify potentially unsafe areas and suggest possibile interventions to solve founded issues. The suggestions can be various and depend on each case. \\ \\
	In the   {\bf AF2}, {\it SafeStreets} allows the {\it Municipality} to generate traffic tickets directly from the application data, derived from the {\it User} notifications'. Additionally, starting from the application data about issued tickets, {\it SafeStreets} can build, looking for trends in the data, statistics and provide insights to the {\it Municipality}. This can help {\it Municipality} to improve the process of issued tickets and understand the effectiveness of {\it SafeStreets}, finding some information useful to improve their services. \\ \\
	From the description above about the purpose of {\it SafeStreets}, we can summarise those goals
		
		\begin{itemize}
   			 \item {\bf [G1]} Allowing {\it Users} to report to the system when a traffic violation occur.		
			 \item {\bf [G2]} Allowing {\it Users} to enter data/information about the violation.
   			 \item {\bf [G3]} Providing both {\it Users} and {\it Authorities} with statistics built from notifications’ data.    			  
   			 \\
			 \\
With regards to {\it {\it Advanced Function 1}}, we identify: 
   			 \item {\bf [G4]} Identifying potentially unsafe areas and making suggestions to address those issues.
   			 \\
			 \\
 With regards to {\it Advanced Function 2}, we identify:
			  \item {\bf [G5]} Allowing {\it Municipality} to generate tickets based on the users’ notifications. 
			  \item {\bf [G6]} Providing statistics built on data from issued tickets to the {\it Municipality}.		
			  \end{itemize}
			
	\subsection{Scope}
	In this section we will distinguish between the {\it Machine}, that's the {\bf S2B} and the {\it World}, that's the portion of the real world affected by the {\it Machine}. This separation, according to the {\it World and Machine} paradigm by Jackson and Zave, leads to a classification of the phenomena in three different types, depending on where they occur. 		
		\subsubsection{World Phenomena}
		World phenomena are events that take place in the real world and do not have a direct impact on the sytem. 
		\begin{itemize}
			\item A generic traffic violation occurs on the streets under the {\it Municipality's} jurisdiction. 
			\item An accident occurs on the streets under the {\it Municipality's} jurisdiction.
			\item A {\it User} notices the violation and takes action.
			\item A ticket is generated by the {\it Municipality}.
			\item Data from all the tickets generated by the {\it Municipality} are stored by the {\it System}.
			\item An intervention is made by the {\it Municipality} to address possible issues on the streets under its jurisdiction.
		\end{itemize}

		\subsubsection{Shared Phenomena}
		Shared phenomena are world phenomena that are shared with the Machine. These are further divided in two categories.\\ \\
		{\bf Controlled by the world and observed by the machine}
		\begin{itemize}
			\item A {\it Guest} signs up on the system giving all the personal data needed and/or signs in with his credentials.
			\item A {\it User} sends a violation notification to the system, including type of violation, position, date, time and picture.
			\item {\it Municipality} transmits data about the accidents occurring on the streets to the {\it System}.
			\item {\it Municipality} evaluates a violation notification coming from the {\it System}. 
			\item {\it Municipality} elaborates a suggestion for an intervention coming from the {\it System}. 
			\item {\it Municipality} transfers data about tickets generated to the {\it System}.
		\end{itemize} 
		{\bf Controlled by the machine and observed by the world}
		\begin{itemize}
			\item The {\it System} notifies the {\it User} that the plate recognition tool was not successful, in order for the user to take a better picture.
			\item The {\it System} transmits a violation notification to the {\it Municipality}.
			\item The {\it System} provides suggestions on possible interventions to address issues to the {\it Municipality}.
			\item The {\it System} shows statistics mined from violation notifications data to {\it Users} and/or the {\it Municipality}.
		\end{itemize}	
		\subsubsection{Machine Phenomena}
		Machine phenomena are events that entirely take place inside the System and cannot be observed in the real world.
		\begin{itemize}
			\item The plate recognition algorithm is ran on the picture of the violation.
			\item The complete set of data are stored and can be retrieved by the {\it System's} DBMS.
		\end{itemize}



	\subsection{Definitions, Acronyms, Abbreviations}
					
		\subsubsection{Definitions}
			
			\begin{itemize}
				\item {\bf Guest:} a user who needs to sign up to the application in order to access its functionalities, otherwise they can't access to the service.
				\item {\bf Authorities:} representatives of the Municipality considered as a kind of application user
				\item {\bf User:} citizen or city's tourist that is registered to the application, so he/she can access the functionality offered by the application. Even if the term User can be considered as a generic term for every application user, in this document we'll consider user as explained above.
				\item {\bf Municipality:} entity composed of both men and the information system that has authority on the streets, where it has the responsibility to enforce the rules and to guarantee safety. 
				\item {\bf Violation:} the action of violating traffic laws.
				\item {\bf Ticket:} administrative sanction established 	by law for a violation.
				\item {\bf Plate recognition Algorithm:} algorithm that 	automatic recognise vehicle's' plate by the images sent from users to report a violation or an accident.
				\item {\bf Notification Data:} information that is provided by the user when he reports a violation. This includes picture, license plate, date, time, position.
				\item {\bf Ticket Data:} information that is provided by the municipality when it adds a new ticket in the system. This includes violation type, license plate, date, time, position.
				\item {\bf Accident Data:} information that usually is 	provided by the municipality when it reports an accident. This can includes accident type, picture, multiple license plate, date, time, position.
				\item {\bf Intervention:} action taken by the municipality to prevent further issues in the city traffic.
				\item {\bf Notification:} message sent by the user to 						advise the system about a violation.
			\end{itemize}
		
		\subsubsection{Acronyms}
		
			\begin{itemize}
				\item {\bf GPS:} Global Positioning System
				\item {\bf API:} Application Programming Interface
				\item {\bf ID:} Identifier
				\item {\bf RASD:} Requirements Analysis and 								Specification Document
				\item {\bf DBMS:} Data Bases Management System
				\item {\bf GDPR:} General Data Protection Regulation
			\end{itemize}

				
				
		\subsubsection{Abbreviations}
		
			\begin{itemize}
				\item {\bf [Gn]:} n-th goal
				\item {\bf [Rn]:} n-th functional requirement
				\item {\bf [Dn]:} n-th domain assumption
				\item {\bf AF1:} Advanced Function One
				\item {\bf AF2:} Advanced Function Two
				\item {\bf SP1:} Shared Phenomena controlled by the 	World and observed by the Machine
				\item {\bf SP2:} Shared Phenomena controlled by the 	Machine and observed by the World
				\item {\bf WP:} World Phenomena
				\item {\bf MP:} Machine Phenomena
			\end{itemize}
			

	\subsection{Revision History}
	
	\begin{table}[ht]
		\centering
		\begin{tabular}{ccc} 
		Version & Date & Changes  \\ 
		\hline
		 \\1.0 & 27/10/2019 & First Delivery
		 \\
		\end{tabular}
		\caption{Revision History}
		\label{default}
	\end{table}
	
	\subsection{Reference Documents}
		
		\begin{itemize}
			 \item Mandatory Project Assignment
			 \item Alloy Official Documentation: http://alloy.lcs.mit.edu/alloy/documentation.html
			 \item ISO/IEC/IEE 29148: System and Software engineering - Life cycle process - Requirements engineering
		\end{itemize}

	\subsection{Document Structure}
			The rest of the document is organised as follows:
				\begin{itemize}
					\item {\bf Overall Description} (Section 2): it will be given a general description of the application, with an analysis of the domain focusing on descriptions about the phenomena according to the {\it World and the Machine} paradigm, and the User Characteristic. It will be provided class diagram and state charts in UML Language and also domain assumptions, dependencies and constraints.
					\item {\bf Specific Requirements} (Section 3): in this section all the Functional Requirements of the application are explained in details and related to use case scenarios and sequence diagrams clarifying process and interactions between the actors and the {\it System}. Also, there are descriptions of all the non-functional requirements and external interfaces.
					\item {\bf Formal Analysis} (Section 4): description and creation of simulation using a formal model, the Alloy specification language in order to address the critical aspects of the {\it SafeStreets System}.				
					\end{itemize}

\pagebreak	
	
%Overall Description
\section{Overall Description}
	\subsection{Product perspective} 
	In this section we will provide a detailed descriptions of World and Shared Phenomena. Also, there  a Class Diagram and State Charts to clarify the modeling of the application. \\ \\
	{\bf A generic traffic Violation occurs on the streets under the \mbox{Municipality's} jurisdiction (WP)} \\
		A violation of the traffic laws occurs on the streets on which the municipality has responsibilities.
Violations includes (but are not limited to) parking where it is forbidden (outside parking lanes, on pedestrian crossings, on sidewalks, exc). \\ \\
	{\bf An Accident occurs on the streets under the Municipality's jurisdiction (WP)} \\
		A car accident occurs on the streets on which the municipality has responsibilities. Because guaranteeing safety is one of those, the municipality aims the number of accidents to be the lowest possible. Car accidents do not only occur between two vehicles, but may also include pedestrians, bikes and urban infrastructure. \\ \\
	{\bf A User notices the violation and takes action (WP)} \\
		A user of the system acknowledges a traffic laws violation and contributes, through the system, to the street safety and rules enforcing effort by reporting the type of violation together with a picture of the vehicle involved. \\ \\
	{\bf A Ticket is generated by the Municipality (WP)}\\
		Representatives of the municipality issue a ticket to the owner of the vehicle involved in a traffic laws violations. \\ \\
	{\bf Data from all the Tickets generated by the Municipality are stored by the System (WP)}\\
		The municipality keeps detailed track in the database of its information system of every ticket issued, including date, time, position, type of violation and plate number.\\ \\
	{\bf An Intervention is made by the Municipality to address possible issues on the streets under its jurisdiction (WP)} \\
		In order to fulfil its responsibility of guaranteeing safety, the municipality makes a modification in the viability or the urban infrastructure layout on the streets under its jurisdiction to address a safety issue.\\ \\
	{\bf The System notifies the User that the plate recognition tool was not successful, in order for the user to take a better picture (SP2)}. \\
		As soon as the system receives from a user a notification of a traffic laws violation, it instantly runs the plate recognition algorithm and in case that fails, the system notifies the user asking him to take a new picture in which the plate is more readable. \\ \\
	{\bf The System transmits a violation notification to the Municipality (SP2)} \\
		The system shares every notification that is reported by users with the municipality, which has the responsibility to take action if necessary.\\ \\
	{\bf The System provides suggestions on possible interventions to address issues to the Municipality (SP2)} \\
		After crossing the information about accidents, which has been transmitted from the municipality, with the data of the violations reported, the system is able to individuate areas and traffic dynamics that are particularly dangerous and to compute possible solutions that would improve safety. \\ \\
	{\bf The System shows statistics mined from violation notifications data to users and/or the Municipality (SP2)} \\
		The system allows both users and municipality to view statistics mined from the notification data. Of course users cannot visualize some of this statistics for privacy reasons (for example the plate of the most egregious violators). \\ \\
	{\bf A Guest signs up on the system giving all the personal data needed and/or signs in with his credentials  (SP1)} \\
		A guest registers on the system sharing his identity, date of birth and email address. He then gets valid credentials to login into the System and use the service. \\ \\
	{\bf A User sends a violation notification to the System, including type of violation, position, date, time and picture (SP1)} \\
		A User reports a violation through the System, taking a picture of the vehicle and selecting the type of violation. The System autonomously includes to that information the date, time and position of the User in the instant he takes the picture. \\ \\
	{\bf Municipality transmits data about the accidents occurring on the streets to the System (SP1)}\\
		The municipality shares complete data about every accident occurring on the streets with the system.\\ \\
	{\bf Municipality evaluates a violation notification coming from the System (SP1)}\\ 
		Representatives of the municipality evaluate a notification transmitted by the system to verify the actual violation of the traffic laws and to generate, if the conditions are right, a ticket. \\ \\
	{\bf Municipality evaluates a suggestion for an intervention coming from the System (SP1)}\\ 
		Once received a suggestion for an intervention from the system, representatives of the municipality evaluate whether it is consistent and feasible and make a cost and benefits analysis.\\ \\
	{\bf Municipality transfers data about tickets generated to the System (SP1)}\\
		The municipality shares complete data about every ticket issued with the system.\\ \\
	{\bf The plate recognition algorithm is ran on the picture of the violation (MP)}\\
		Once received the picture taken by the user, the system runs his tool to read the plate number directly from the picture, to integrate the set of data of the notification.\\ \\
	{\bf The complete set of data are stored and can be retrieved by the System's DBMS (MP)}\\
		The system’s DBMS assures integrity and durability of every piece of data managed by the system: every notification, information on issued tickets and statistics about the previous are available to be retrieved with a proper query.\\
		
		
	\subsubsection{Class Diagrams}
		

	\subsection{Product functions}
	\begin{itemize}
	\item {\bf User functions:} \\
	
	
		{\bf Report a traffic violation} \\
		The User noticing a violation on the streets is able to, once accessed the System, fill in the information to report a violation: he takes a picture of the vehicle, with the plate decently readable, and select the type of violation from the list provided. The date, time and position are automatically taken by the system when the image is shot (the picture of the violation can only be taken using the application, not retrieved and uploaded from the photo archive) and integrated to the information given by the User. All the data is eventually sent to the System’s DBMS and the plate recognition algorithm is ran to complete the data with the plate number. If the tool fails, the User is instantly notified and asked to take a better picture.\\
	
		{\bf Access statistics from notifications’ data}\\
		The User, once logged in, is able to enter the section of the application where statistics from the notifications received are available: number of violations per period (week, month, year), violations average per day of the week, the previous two per type of violation and a map showing where the violations were reported. Of course no plate number and no information about who reported the violations is shown to the User.\\
	
	\item {\bf Municipality functions:} \\
	
		{\bf Access statistics from notifications’ data}\\
		The Municipality is able to enter the section of the system interface where statistics from the notifications received from its jurisdiction are available: number of violations per period (week, month, year), violations average per day of the week, the previous two per type of violation and a map showing where the violations were reported. The municipality can view the plate number of the offender of any violation notification but has no information on the User that reported it. \\
		
		{\bf Obtain suggestions about unsafe areas}\\
		The system provides to the Municipality possible solutions to issues individuated through statistics, which are being mined from data on accidents and violations. This is something really valuable for the Municipality because most of the time it doesn’t have the tools itself to individuate connections and cause-effect relationship in the series of events (ex. a lot of accidents happening at a corner where a lot of second row parking violations are notified).\\
		
		{\bf Generate tickets}\\
		The Municipality receives every notification of violations occurring under its jurisdiction and evaluates every one of them to determine whether a violation actually occurred, if the type of violation reported is correct and if there are the conditions to issue a ticket, in which case the ticket generated is no different from any ticket coming from a physically-verified violation. By exploiting the System’s information the municipality is able to issue a much larger number of tickets in its jurisdiction and to gain effectiveness in enforcing the rules. Every information about generated tickets is then shared with the System.\\
		
		{\bf Access statistics from issued tickets}\\
		The Municipality is able to enter the section of the system interface where statistics from the tickets generated are available: number of tickets per period (week, month, year), tickets average per day of the week, the previous two per type of violation sanctioned, a map showing where the violations were reported.\\
		
		
	\end{itemize}
		
	\subsection{User characteristics}
	{\it SafeStreets} has two kinds of users. There are  {\bf simple }{\it Users} of the application and {\it Authorities} (for details about the semantics about the term User refers to the section 1.3.1 of this document). The first type is usually a citizen or a tourist of the city where the {\it SafeStreets'} service is activated. The {\it User} needs to notify authorities of a violation,  and in order to do this, he/she has to install the application on his/her smartphone. The second type of user are {\it Authorities}, they are representative of the {\it Municipality}. They can access the application from their smartphone or any device supplied by the {\it Municipality}. Also their access to the service is provided by the {\it Municipality} too because they have a different level of visibility in respect to the simple {\it Users}. They need to access statistics provided by the service or have to generate tickets. {\bf (? da rivedere se vogliamo tenere questa frase, in base a come decidiamo di gestire la generazione del ticket)}. \\
	The characters involved are:
	\begin{itemize}
		\item {\it Guest:} a user who has downloaded the application on his/her smartphone, but has to create an account. No functionalities can be accessed without registering to the service
		\item {\it User:} a user who has created an account, but has to log-in to access the feature. Signing-up with all the necessary data or inserting his credentials, he/she can access all the functionalities offered by the application.
		\item {\it Logged-in User:} a user who has logged-in the application providing the credentials established during the registration phase. He's able to access all the functionalities. 
		\item {\it Authorities:} a user that's a representative of the Municipality, with its own privileges and levels of visibility. Logging-in with his/her credentials, provided directly from the Municipality, there's possibility to access the functionalities.
		\item {\it Logged-in Authorities:} a user that's a representative of the Municipality, that has provided the credentials and can access all the functionalities.
	\end{itemize}
	
	\subsection{Assumptions, dependencies and constraints}
		\subsubsection{Constraints}
		\begin{itemize}
		\item {\it Users} are located on the streets under the {\it Municipality} jurisdiction when they notify the violation.
		\item {\it Users} must have a smartphone application to access the {\it System}.
		\item The {\it System} must respect the GDPR regulation.
		\item The {\it System} must ask the permission to process personal data about {\it Users}.
		\item The {\it System} must guarantee that the information is never altered during the whole process. 
		\end{itemize}
		\subsubsection{Dependencies}
		\begin{itemize}
		\item The {\it System} needs a DBMS service to retrieve and store data.
		\item The {\it System} make use of the GPS provided by the smartphone.
		\item The {\it System} uses a map visualisation service.
		\item The {\it System} make use of internet connection provided by the smartphone.
		\end{itemize}
		
		\subsubsection{Domain Assumptions}
		\begin{itemize}
		 \item {\bf [D1]} Each user, simple one or authority, has only one account.
		 \item {\bf [D2]} Data provided during the registration are correct and belong to the person who created the account
		 \item {\bf [D3]} A violation should be clearly visible and identifiable from the picture.
		 \item {\bf [D4]} A violation is processed if everything is correct, otherwise the Municipality fixes the errors.
		 \item {\bf [D5]} Data from Municipality about incidents or issued tickets are correct.
		 \item {\bf [D6]} GPS system of the device has an accuracy of 50 meters.
		 \item {\bf [D7]} Permission to access GPS and device data is granted to the System.
		 \item {\bf [D8]} Each car has a license plate.
		 \item {\bf [D9]} Municipality has the ownership to issue tickets even if the violation is not physically acknowledged.
		 \item {\bf [D10]} Suggestions from the System are not discarded from the Municipality, but they are evaluated as a feasible solution.
		 \end{itemize}
		
\pagebreak	


	
\end{document}